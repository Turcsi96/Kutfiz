\documentclass{article}
\usepackage[T1]{fontenc}
\usepackage[magyar]{babel}
\usepackage[utf8]{inputenc}
\usepackage{graphicx}

\pagestyle{plain}
\title{\textbf{Euler módszer}}
\author{Tuhári Richárd}
\date{}

\begin{document}

\maketitle

\section*{\textbf{Rövid leírás}}

Alábbiakban a címben említett Euler módszer lesz bemutatva. Jelen esetben a koordinátarendszerben végig (0,0) pontban elhelyezkedő Föld körül keringő Holdra lesz alkalmazva.

\begin{center}
\begin{figure}[h!]
\includegraphics{build/res.png}
\end{figure}
\end{center}

\newpage

Mint fentebb látható, hogy az Euler módszer nem a legjobb, mivel a rendszer nyeri az energiát. A képre van plotolva az analítikus megoldás is, így az eltérés még jobban látszik.

\end{document}
